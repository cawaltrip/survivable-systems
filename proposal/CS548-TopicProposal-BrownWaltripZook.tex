\documentclass[twoside, conference]{IEEEtran} 

\usepackage[usenames,dvipsnames]{color}
\usepackage{hyperref}
\usepackage[nameinlink]{cleveref}
%\usepackage{cite}

% PDF Metadata, Link Coloring
%\hypersetup{
%	pdftitle={CS548: Survivable Systems\\Semester Project Proposal Draft 1},
%	pdfauthor={Matt Brown; Chris Waltrip; Jared Zook},
%	pdfsubject={University of Idaho; CS-520: Graduate Paper},
%	pdfcreator={Chris Waltrip},
%	pdfproducer={Chris Waltrip},
%	linktoc=all, 			% Link the section number, text and page number in Contents
%	colorlinks=false,        % Removes color frame and colors text instead
%	linkcolor=ForestGreen,  % Default is red, may want to use black
%	citecolor=Bittersweet,  % Default is green
%	filecolor=Cyan,         % Default is cyan
%	urlcolor=Magenta,       % Default is magenta
%}

% Title Information
\title{CS-548: Survivable Systems\\Semester Project Proposal Draft 1}
\author{
	\IEEEauthorblockN{Matt Brown}
	\IEEEauthorblockA{Department of Computer Science\\University of Idaho\\Moscow, Idaho 83843\\Email: \href{mailto:matt2714@vandals.uidaho.edu}{\nolinkurl{matt2714@vandals.uidaho.edu}}}
	\and
	\IEEEauthorblockN{Chris Waltrip}
	\IEEEauthorblockA{Department of Computer Science\\University of Idaho\\Moscow, Idaho 83843\\Email: \href{mailto:walt2178@vandals.uidaho.edu}{\nolinkurl{walt2178@vandals.uidaho.edu}}}
	\and
	\IEEEauthorblockN{Jared Zook}
	\IEEEauthorblockA{Department of Computer Science\\University of Idaho\\Moscow, Idaho 83843\\Email: \href{mailto:jzook@vandals.uidaho.edu}{\nolinkurl{jzook@vandals.uidaho.edu}}}
}

\begin{document}
\maketitle

\begin{abstract}
	This paper is a survey of recent research that addresses the problem of sensor failure in autonomous vehicles. Modern mathematical and fault models are presented, and their strengths and weaknesses are assessed. In particular, we consider the different faults that should be assumed, as well as any real-time constraints that are imposed in the environment. This initial draft of our semester project includes highlights from information we have found so far and provides a roadmap for what direction we intend to go with our research.


include summary of direction here
\end{abstract}

\section{Introduction}

	write after individual sections completed [jZook]

\section{Research Highlights}

\subsection{Matt}
Unmanned vehicles have a much higher sensor count then manned vehicles. These sensors include proximity sensors, RADAR, LIDAR, and GPS. These sensors are deployed in networks that have to agree on input values so they can be process accurately. In order for these sensors to agree they need to be implemented redundantly. An agree method adapted from Yansong Ren and associates can be applied to this situation. In the algorithm proposed by Yansong Ren and associates they describe a set of redundant nodes that communicate through locally before broadcasting the results. If we adapt this algorithm to a real world sensor net the, a physically local group of sensors can be treated as a redundant group. This redundant group can agree on their readings via a majority algorithm or median agreement algorithm, to remove the readings of nodes in the group that read nothing. Once the sensor nodes have an agreed value they can pass the information to the main computer or processing module via multicast so all redundant nodes and the main processing module receive the same value. Through this method the transmitting node is held accountable to the value agreed.\cite{Ren2001}

Ren's paper also describes a method for removing or replacing faulting nodes. This method includes a manager that checks the value of the sent from the transmitting node to check for faulting. This method describes ``replacing" faulting nodes with new nodes. In a unmanned vehicle this method is not very practical. This idea can be adapted, however, to a module that can detect faulting modules that ``pushes" new copies of firmware to a faulting sensor in an attempt to repair it. This method fails however if a sensor is physically damaged.\cite{Ren2001}

Another aspect of redundancy that should be addressed is faulting nodes due to a failure in communication. If the communication system in a unmanned vehicle is designed to be self-repairing, through redundancy, it can take some time for the module to notice the failing path. In this case there needs to be a method for reintegrating sensors once communication is reestablished. A paper written by K.-Q. Yan and S.-C. Wang describes an agreement algorithm that involves nodes that are entering and exiting communication channels \cite{Yan2007}. We believe this algorithm can be adapted to the scenario that involves a sensor node losing contact with the rest of its redundancy group. In the algorithm described in \cite{Yan2007} one could implement a method for reestablishing contact with a sensor node even while in the middle of voting. During normal operation under this algorithm, all nodes that start a round of voting, nodes collect data and pass all there data to all other nodes in the group. All nodes in the group then exchange their values and then vote internally on the values. The nodes then pass exchange the results of there voting and vote once more. After the second voting ``session" all nodes should be agreeing on the same value. According to the algorithm, if a node loses contact it is no longer included in voting for that round unless it reconnects before the end of the second voting ``session". If a node reconnects before the second voting algorithm all other nodes all pass the results of the second exchange to the new node(s). Once all data is transmitted the second ``session" of voting occurs and at this point all values should have the same value \cite{Yan2007}. This algorithm should work for nodes that temporarily lose contact to the group without the group losing accuracy.

\subsection{Chris}
Sensor Network Data Fault Types paper\cite{Ni2009}.  With respect to event detection, an event may appear as a fault if the reading is outside of the expected range \cite[p. 25:5]{Ni2009}.  Use common fault datasets and the sensor networks application to help determine fault detection algorithm design \cite[p. 25:7]{Ni2009}.  Prognostic and Health Management (PHM) defined as ``\textit{a system engineering discipline focusing on detection, prediction, and management of the health and status of complex engineered systems}'' \cite[p. 1]{Ma2009}.  Looking at the idea of transmissive asymmetric faults while monitoring a sensor network (the idea of sensors "lying") \cite[p. 4]{Ma2009}.  Modeling real-time fault tolerance using dynamic fault models \cite[p. 4]{Ma2009}\cite{Ma2008}.  How evolutionary game theory is used in regards to these dynamic hybrid fault models \cite[p. 6]{Ma2009} and Byzantine generals playing evolutionary games \cite[p. 11,15]{Ma2009}.  One thing to do is look at how these models line up with some of the models with some of the hybrid fault models that we've looked at in class \cite{Azadmanesh2000}.  Another thing that I'd like to look at is consensus when link failures exist \cite{Biely2011} and approximate agreement in partially connected networks \cite{Srinivasan2007}.


\subsection{Wireless Network Fault Models [jZook]}

% ultimately boil turquoise down into a single long paragraph, placing
% traditional fault model criticisms elsewhere

One protocol specifically designed for use in wireless ad hoc networks is Turquois, an ``asynchronous Byzantine consensus protocol'' \cite{Moniz2013}{den}. Turquois was developed because the authors found that traditional fault models did not map well to the pervasive communication failures common to wireless ad hoc networks (e.g. ad hoc networks created and utilized by semi-autonomous vehicles in transit) \cite{Moniz2013}. Traditional models, such as the Lamppost model assumed general reliability in terms of the \textit{links} of the different nodes in a system \cite{Moniz2013}. That applies well enough to standard wired networks, however, the nature of wireless communication and the dynamic properties of mobile ad hoc networks, give rise to unreliable connections due to, for example, interference, malicious jamming, and node mobility \cite{Moniz2013}. Thus, it is necessary to develop expanded models that will be better fit to the contingencies in newer applications. As such, Turquois assumes an ad hoc, dynamic fault\cite{Moniz2013}.

Turquois finds agreement in the presence of two faults: dynamic transmission omission faults (ones in which a message sent by $P_i$ is not received by $P_j$) and Byzantine faults (asymmetric faults such as collisions or false values sent) \cite{Moniz2013}. Its aim is to operate \textit{even with} the assumption that some messages are guaranteed to be permanently corrupted or lost on a transitory basis \cite{Moniz2013}. Like the algorithms in \cite{bialy}, Turquois round-based, with messages being transmitted each round, and the presence of faults is accounted for in the assumption \cite{Moniz2013}. Accordingly, Turquois is able to tolerate dynamic message omissions and Byzantine faults and is safe despite the failure of $f < \frac{n}{3}$ nodes \cite{Moniz2013}. Similar to the HO model mentioned in \cite{bialy}, agreement is reached under this protocol when the conditions of Validity, Agreement, and Termination are met \cite{Moniz2013}.

Wireless ad hoc networks are inherently resource-constrained, especially when compared with static systems \cite{Moniz2013}. As such, computational expense is something implementers need to pay special attention to. To address this with the Turquois protocol, the authors eschewed the use of RSA-based public-key cryptography, which is computationally expensive \cite{Moniz2013}. Instead, they used hash-based message authentication and claimed to achieve an entire magnitude of performance when scaling their simulation to have a high number of nodes and comparing it to a public-key solution \cite{Moniz2013}. There was no mention about whether or not eliminating RSA compromised security in any way.

In \cite{Ma2008}, traditional hybrid fault models are laid as the ``mathematical foundations of redundancy management in fault tolerance''. However, they fall short when applied to dynamic systems such as WSNs, where nodes may fail at different times \cite{Ma2008}. WSNs, then, require analysis of real-time processing mechanics, and consideration of other contingencies unique to the constraints of their environment \cite{Ma2008}. Traditional hybrid fault models such as \cite{azad} do not account for timing of faulty behavior, for example, so they neglect to consider the situation that a fault seen at $t_0$ may transform into some other type of fault at $t_1$ \cite{Ma2008}. To address this gap and to assume a fault model for WSNs, \cite{Ma2008} assumes a dynamic fault model that includes a time-dependent failure rate as well as time-dependent failure modes \cite{Ma2008}. The fault classifications from the traditional models are then seen as \textit{qualitative} parameters which may then be used as strategies in Evolutionary Game Theory (EGT) games \cite{Ma2008}.

Since traditional reliability analysis is declared qualitative, an alternative fault model must be assumed. \cite{Ma2008} suggests the addition of \textit{survivability analysis} to the traditional process which allows for consideration of time and covariant-dependent failures. To do this, survivor functions are included with agreement algorithms \cite{Ma2008}. Then, EGTs are used to simulate the reliability aspect of the model \cite{Ma2008}. In the EGT, the qualitative categories from the traditional fault models serve as the constraints to the model, and the Byzantine generals serve as the players of the game \cite{Ma2008}. When conducting EGTs, we hope to arrive at an Evolutionary Stable Strategy (ESS); finding the ESS leads us to a payoff: reliability \cite{Ma2008}. 

\bibliographystyle{IEEEtranS}
\bibliography{../CS548-Bibliography}

\end{document}























