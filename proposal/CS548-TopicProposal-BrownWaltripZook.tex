\documentclass[twoside, conference]{IEEEtran} 

\usepackage[usenames,dvipsnames]{color}
\usepackage{hyperref}
\usepackage[nameinlink]{cleveref}
\usepackage{cite}

% PDF Metadata, Link Coloring
%\hypersetup{
%	pdftitle={CS548: Survivable Systems\\Semester Project Proposal Draft 1},
%	pdfauthor={Matt Brown; Chris Waltrip; Jared Zook},
%	pdfsubject={University of Idaho; CS-520: Graduate Paper},
%	pdfcreator={Chris Waltrip},
%	pdfproducer={Chris Waltrip},
%	linktoc=all, 			% Link the section number, text and page number in Contents
%	colorlinks=false,        % Removes color frame and colors text instead
%	linkcolor=ForestGreen,  % Default is red, may want to use black
%	citecolor=Bittersweet,  % Default is green
%	filecolor=Cyan,         % Default is cyan
%	urlcolor=Magenta,       % Default is magenta
%}

% Title Information
\title{CS-548: Survivable Systems\\Semester Project Proposal Draft 1}
\author{
	\IEEEauthorblockN{Matt Brown}
	\IEEEauthorblockA{Department of Computer Science\\University of Idaho\\Moscow, Idaho 83843\\Email: \href{mailto:matt2714@vandals.uidaho.edu}{\nolinkurl{matt2714@vandals.uidaho.edu}}}
	\and
	\IEEEauthorblockN{Chris Waltrip}
	\IEEEauthorblockA{Department of Computer Science\\University of Idaho\\Moscow, Idaho 83843\\Email: \href{mailto:walt2178@vandals.uidaho.edu}{\nolinkurl{walt2178@vandals.uidaho.edu}}}
	\and
	\IEEEauthorblockN{Jared Zook}
	\IEEEauthorblockA{Department of Computer Science\\University of Idaho\\Moscow, Idaho 83843\\Email: \href{mailto:jzook@vandals.uidaho.edu}{\nolinkurl{jzook@vandals.uidaho.edu}}}
}

\begin{document}
\maketitle

\begin{abstract}
	This paper is a survey of recent research that addresses the problem of sensor failure in autonomous vehicles. Modern mathematical and fault models are presented, and their strengths and weaknesses are assessed. In particular, we consider the different faults that should be assumed, as well as any real-time constraints that are imposed in the environment. This initial draft of our semester project includes highlights from information we have found so far and provides a roadmap for what direction we intend to go with our research.


include summary of direction here
\end{abstract}

\section{Introduction}

	write after individual sections completed [jZook]

\section{Research Highlights}

\subsection{Matt}
Unmanned vehicles have a much higher sensor count then manned vehicles. These sensors include proximity sensors, RADAR, LIDAR, and GPS. These sensors are deployed in networks that have to agree on input values so they can be process accurately. In order for these sensors to agree they need to be implemented redundantly. An agree method adapted from Yansong Ren and associates can be applied to this situation. In the algorithm proposed by Yansong Ren and associates they describe a set of redundant nodes that communicate through locally before broadcasting the results. If we adapt this algorithm to a real world sensor net the, a physically local group of sensors can be treated as a redundant group. This redundant group can agree on their readings via a majority algorithm or median agreement algorithm, to remove the readings of nodes in the group that read nothing. Once the sensor nodes have an agreed value they can pass the information to the main computer or processing module via multicast so all redundant nodes and the main processing module receive the same value. Through this method the transmitting node is held accountable to the value agreed.\cite{Ren2001}

Ren's paper also describes a method for removing or replacing faulting nodes. This method includes a manager that checks the value of the sent from the transmitting node to check for faulting. This method describes ``replacing" faulting nodes with new nodes. In a unmanned vehicle this method is not very practical. This idea can be adapted, however, to a module that can detect faulting modules that ``pushes" new copies of firmware to a faulting sensor in an attempt to repair it. This method fails however if a sensor is physically damaged.\cite{Ren2001}

Another aspect of redundancy that should be addressed is faulting nodes due to a failure in communication. If the communication system in a unmanned vehicle is designed to be self-repairing, through redundancy, it can take some time for the module to notice the failing path. In this case there needs to be a method for reintegrating sensors once communication is reestablished. A paper written by K.-Q. Yan and S.-C. Wang describes an agreement algorithm that involves nodes that are entering and exiting communication channels \cite{Yan2007}. We believe this algorithm can be adapted to the scenario that involves a sensor node losing contact with the rest of its redundancy group. In the algorithm described in \cite{Yan2007} one could implement a method for reestablishing contact with a sensor node even while in the middle of voting. During normal operation under this algorithm, all nodes that start a round of voting, nodes collect data and pass all there data to all other nodes in the group. All nodes in the group then exchange their values and then vote internally on the values. The nodes then pass exchange the results of there voting and vote once more. After the second voting ``session" all nodes should be agreeing on the same value. According to the algorithm, if a node loses contact it is no longer included in voting for that round unless it reconnects before the end of the second voting ``session". If a node reconnects before the second voting algorithm all other nodes all pass the results of the second exchange to the new node(s). Once all data is transmitted the second ``session" of voting occurs and at this point all values should have the same value \cite{Yan2007}. This algorithm should work for nodes that temporarily lose contact to the group without the group losing accuracy.

\subsection{Chris}

\subsection{Jared}

\bibliographystyle{IEEEtranS}
\bibliography{../CS548-Bibliography}

\end{document}























